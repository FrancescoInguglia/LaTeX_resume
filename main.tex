%%%%%%%%%%%%%%%%%
% This is an example CV created using altacv.cls (v1.1, 21 November 2016) written by
% LianTze Lim (liantze@gmail.com), based on the 
% Cv created by BusinessInsider at http://www.businessinsider.my/a-sample-resume-for-marissa-mayer-2016-7/?r=US&IR=T
% 
%% It may be distributed and/or modified under the
%% conditions of the LaTeX Project Public License, either version 1.3
%% of this license or (at your option) any later version.
%% The latest version of this license is in
%%    http://www.latex-project.org/lppl.txt
%% and version 1.3 or later is part of all distributions of LaTeX
%% version 2003/12/01 or later.
%%%%%%%%%%%%%%%%

%% If you want to use \orcid or the
%% academicons icons, add "academicons"
%% to the \documentclass options. 
%% Then compile with XeLaTeX or LuaLaTeX.
% \documentclass[10pt,a4paper,academicons]{altacv}
\documentclass[10pt,a4paper]{altacv}

%% AltaCV uses the fontawesome and academicon fonts
%% and packages. 
%% See texdoc.net/pkg/fontawecome and http://texdoc.net/pkg/academicons for full list of symbols.
%% When using the "academicons" option,
%% Compile with LuaLaTeX for best results. If you
%% want to use XeLaTeX, you may need to install
%% Academicons.ttf in your operating system's font %% folder.


% Change the page layout if you need to
\geometry{left=1cm,right=9cm,marginparwidth=6.8cm,marginparsep=1.2cm,top=1cm,bottom=1cm}

% Change the font if you want to.

% If using pdflatex:
\usepackage[utf8]{inputenc}
\usepackage[T1]{fontenc}
\usepackage[default]{lato}
\usepackage{setspace}
\usepackage{hyperref}
\hypersetup{colorlinks=true, urlcolor=blue}

% If using xelatex or lualatex:
% \setmainfont{Lato}

% Change the colours if you want to
\definecolor{DarkBlue}{HTML}{145A32}
\definecolor{Black}{HTML}{000000}

\colorlet{heading}{DarkBlue}
\colorlet{accent}{DarkBlue}
\colorlet{emphasis}{Black}
\colorlet{body}{Black}

% Change the bullets for itemize and rating marker
% for \cvskill if you want to
\renewcommand{\itemmarker}{{\small\textbullet}}
\renewcommand{\ratingmarker}{\faCircle}


%% sample.bib contains your publications
\addbibresource{sample.bib}

\begin{document}
\name{Francesco Inguglia}
  \tagline{Ingegnere Gestionale}
% Cropped to square from https://en.wikipedia.org/wiki/Marissa_Mayer#/media/File:Marissa_Mayer_May_2014_(cropped).jpg, CC-BY 2.0
% foto circolare
\photo{5cm}{sagoma}




\personalinfo{%
  % Not all of these are required!
  % You can add your own with \printinfo{symbol}{detail}
  \doublespacing
  \email{\href{mailto:francescoinguglia96@gmail.com}{francescoinguglia96@gmail.com}}
  \linkedin{\href{https://www.linkedin.com/in/francesco-inguglia/}{Francesco Inguglia}}
  \phone{+39 3200270194}
  \present{05/07/1996}
%   \phone{000-00-0000}
%  \mailaddress{20032, Cormano, MI}
  \location{Italy}
%  \car{Class B}
%  \homepage{}
% I'm just making this up though.
%   \orcid{orcid.org/0000-0000-0000-0000} % Obviously making this up too. If you want to use this field (and also other academicons symbols), add "academicons" option to \documentclass{altacv}
}



%% Make the header extend all the way to the right, if you want. Extend the right margin by 8cm (=6.8cm marginparwidth + 1.2cm marginparsep)
\begin{adjustwidth}{}{-8cm}
\makecvheader
\end{adjustwidth}

%% Provide the file name containing the sidebar contents as an optional parameter to \cvsection.
%% You can always just use \marginpar{...} if you do
%% not need to align the top of the contents to any
%% \cvsection title in the "main" bar.
\cvsection[page1sidebar]{About Me}
Mi presento: sono \textbf{Francesco, Neolaureato in Ingegneria Gestionale} con una profonda passione per la tecnologia. Nutro un grande
interesse per il \textbf{settore automotive} e per il \textbf{settore dell'energia elettrica}. Sono costantemente alla ricerca di
opportunità per arricchire il mio bagaglio culturale attraverso nuove esperienze. L'innovazione e l'acquisizione di
nuove conoscenze rappresentano due pilastri fondamentali della mia forma mentis. Inoltre, mi considero un
individuo proattivo e incline al lavoro di squadra, partecipando sempre in modo attivo e determinato alle sfide
che mi vengono presentate.


%\begin{document}
%    \begin{figure}
%        \centering
%            \includegraphics[width=0.30\textwidth]{Exito1.jpg}
%            \centering
%    \end{figure}

% \divider

% \cvevent{Product Engineer}{Google}{23 June 1999 -- 2001}{Palo Alto, CA}

% \begin{itemize}
% \item Joined the company as employe \#20 and female employee \#1
% \item Developed targeted advertisement in order to use user's search queries and show them related ads
% \end{itemize}

\cvsection{EXPERIENCE}

\cvevent{Tirocinante Business Analyst}{Elmi s.r.l.}{6 Nov 2023 - 30 Nov 2023}{Palermo}
%    \begin{itemize}
%        \item I managed \textbf{Big Data flows} for a financial client, from data \textbf{ingestion} to \textbf{scheduling} and data                                   \textbf{transformation}, speeding up the build update time by approximately \textbf{50\% }and significantly \textbf{reducing}                both \textbf{computational costs} and \textbf{data storage space}.
%    \end{itemize}
%   \begin{itemize}
%        \item I developed a complex application that allows for the calculation and scheduling of risk indicators. I implemented the \textbf{backend} using \textbf{Python} and the \textbf{frontend} with \textbf{React}. This application integrates seamlessly with a \textbf{forecasting model} for these indicators, enabling the client to \textbf{anticipate market trends} and take \textbf{proactive} or \textbf{corrective} actions.
 %   \end{itemize}
    
 %   \begin{itemize}
 %       \item I oversaw the development of a \textbf{clustering algorithm} designed to \textbf{identify users} with \textbf{anomalous permissions} within the client's database. This initiative resulted in \textbf{reducing the auditor's workload} by more than \textbf{fivefold} and significantly \textbf{improving} the \textbf{precision} of anomalous user detection.
   %     \item Hold several hours of lectures in a collaboration with \textbf{\href{https://www.h-farm.com/it/education/college/master-universitari}{H-Farm}}, \textbf{teaching} the fundamentals of \textbf{Machine Learning} and \textbf{Big Data}.
 %   \end{itemize}
Durante il mio periodo di tirocinio, ho condotto un'analisi approfondita del \textbf{Decreto Legge 81/2008}. Ho sviluppato
 una presentazione in formato PowerPoint per esporre in maniera chiara i dati raccolti. Successivamente, ho
 elaborato i dati rilevanti tramite tabelle Excel che sono stati caricati in seguito sulla piattaforma \textbf{IBM Maximo}.
 L'obiettivo di questo progetto consiste nella creazione di un applicativo da presentare e proporre alle aziende, al
 fine di monitorare e gestire i molteplici adempimenti previsti all'interno del decreto legislativo in questione.


\cvsection{Training Courses}
\begin{itemize}
    \item \textbf{Power BI Desktop} per l'analisi e la visualizzazione dei dati
    \item Impariamo da zero \textbf{SQL con Oracle, SQL Server e MySQL}
    \item Data Analisi con \textbf{Pandas}
    \item \textbf{Tableau} per l'analisi dei dati
\end{itemize}
\cvsection{DIGITAL SKILLS}
\begin{itemize}
  \item Padronanza del Pacchetto Office \textbf{(Word Excel PowerPoint ecc)}
  \item Esperienza con piattaforme digitali \textbf{(Teams, Skype, Google Meet, Zoom)}
  \item Conoscenza piattaforme cloud online \textbf{(Drive, OneDrive, iCloud)}
  \item Buon uso di sistemi informatici base \textbf{(Windows, MacOS, Linux)}
\end{itemize}

%\clearpage

%\begin{document}
%    \begin{figure}
%        \centering
%            \includegraphics[width=0.30\textwidth]{Exito1.jpg}
%            \centering
%    \end{figure}

%\end{document}

\end{document}
